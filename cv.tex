% resume.tex
% vim:set ft=tex spell:

\documentclass[10pt,letterpaper]{article}
\usepackage[letterpaper,margin=0.75in]{geometry}
\usepackage[utf8]{inputenc}
\usepackage{mdwlist}
\usepackage[T1]{fontenc}
\usepackage{textcomp}
\usepackage{tgpagella}
\usepackage{latexsym}
\usepackage{amssymb}
\usepackage{hyperref}
\pagestyle{empty}
\setlength{\tabcolsep}{0em}

% indentsection style, used for sections that aren't already in lists
% that need indentation to the level of all text in the document
\newenvironment{indentsection}[1]%
{\begin{list}{}%
	{\setlength{\leftmargin}{#1}}%
	\item[]%
}
{\end{list}}

% opposite of above; bump a section back toward the left margin
\newenvironment{unindentsection}[1]%
{\begin{list}{}%
	{\setlength{\leftmargin}{-0.5#1}}%
	\item[]%
}
{\end{list}}

% format two pieces of text, one left aligned and one right aligned
\newcommand{\headerrow}[2]
{\begin{tabular*}{\linewidth}{l@{\extracolsep{\fill}}r}
	#1 &
	#2 \\
\end{tabular*}}

% make "C++" look pretty when used in text by touching up the plus signs
\newcommand{\CPP}
{C\nolinebreak[4]\hspace{-.05em}\raisebox{.22ex}{\footnotesize\bf ++}}

% and the actual content starts here
\begin{document}

\begin{center}
{\LARGE \textbf{Sebastian Ruder}}

\url{ruder.io} \textbullet
\ \ \texttt{sebastian@ruder.io}
\\
+353 89 237 9192 \ \ \textbullet
\ \ 9 New Bride Street\ \ \textbullet
\ \ Dublin, Ireland
\end{center}

\hrule
\vspace{-0.4em}
\subsection*{Experience}

\begin{itemize}
	\parskip=0.1em
	
	\item
	\headerrow
		{\textbf{\href{http://www.aylien.com/}{AYLIEN}}}
		{\textbf{Dublin, Ireland}}
	\\
	\headerrow
		{\emph{Research Scientist}}
		{\emph{10/2015 -- Present}}
	\begin{itemize*}
		\item Developed aspect-based sentiment analysis (ABSA) endpoint\footnote{\url{https://developer.aylien.com/text-api-demo?tab=absa}} and created sentiment analysis on-par with state-of-the-art\footnote{\url{https://developer.aylien.com/text-api-demo?tab=sentiment}}.
		\item My current work focuses on developing efficient domain adaptation and semi-supervised learning algorithms as well as developing state-of-the-art models for applications such as stance detection and emotion detection.
	\end{itemize*}

	\item
	\headerrow
		{\textbf{IBM}}
		{\textbf{Munich, Germany}}
	\\
	\headerrow
		{\emph{Extreme Blue Intern, Watson}}
		{\emph{08/2015 -- 09/2015}}
	\begin{itemize*}	
		\item Design and implementation of text analysis ML components applied to customer data of leading German insurance company \emph{Versicherungskammer Bayern}; automatically identifies structural semantics and sentiment of incoming e-mails, e.g. complaints and classifies email based on reason for complaint.
		\item Pitched project to audience at European Expo and was chosen as one of eight teams to pitch to IBM customers; project was referred to as a "lighthouse project for Watson in Europe" by jury members.
		\item Project was awarded Digital Thought Leadership award in leading contest of German insurance industry by leading German newspaper \emph{Süddeutsche Zeitung} and Google\footnote{\url{https://www.sv-veranstaltungen.de/site/fachbereiche/versicherungs-leuchtturm}} and covered by \emph{Süddeutsche Zeitung}\footnote{\url{http://www.sueddeutsche.de/wirtschaft/kuenstliche-intelligenz-aerger-fuer-watson-1.2772927}}.
	\end{itemize*}

	\item
	\headerrow
		{\textbf{Microsoft}}
		{\textbf{Dublin, Ireland}}
	\\
	\headerrow
		{\emph{Linguistic Engineering Intern, XBox}}
		{\emph{02/2015 -- 06/2015}}
	\begin{itemize*}
		\item Contributed to developing an ML system for analyzing linguistic complexity of strings in C\# for localization prioritization during testing; performed feature analysis and framed problem as anomaly detection.
		\item Created proof of concept and implemented morphology-based terminology validation algorithm.
		\item Evangelized customer sentiment analysis efforts, drove cross-team collaboration, and provided insights to stakeholders.
	\end{itemize*}

	\item
	\headerrow
		{\textbf{The OpenCog Foundation}}
		{\url{opencog.org}}
	\\
	\headerrow
		{\emph{Google Summer of Code Intern}}
		{\emph{Summer 2014}}
	\begin{itemize*}
		\item Enabled system to make common-sense inferences using deductive reasoning, e.g. \emph{All men are mortal. Socrates is a man.} $\rightarrow$ \emph{Socrates is mortal.}
		\item Applied inference using probabilistic logic networks on the output of a relationship extractor.
		\item Documented and extended Python code for temporal inference.
	\end{itemize*}

	\item
	\headerrow
		{\textbf{Lingenio GmbH}}
		{\textbf{Heidelberg, Germany}}
	\\
	\headerrow
		{\emph{Software Engineering Intern}}
		{\emph{Spring 2014}}
	\begin{itemize*}
		\item Created a converter from TBX to Lingenio native format and vice versa.
		\item Integrated TBX term bases in Dictionary Server; created localized web service using Jinja2, Flask-Babel, and lighttpd.
	\end{itemize*}

	\item
	\headerrow
		{\textbf{SAP}}
		{\textbf{Walldorf, Germany}}
	\\
	\headerrow
		{\emph{Working Student, Development University}}
		{\emph{02/2013 -- 02/2014}}
	\begin{itemize*}
		\item Created content for internal programming and Design Thinking courses.
		\item Automated reporting processes, e.g. reduced expenditure of work for monthly training report by > 75\%, i.e. from 8 hours to 2 hour using Excel / VBA scripts.
	\end{itemize*}

	\item
	\headerrow
		{\textbf{TEMIS}}
		{\textbf{Heidelberg, Germany}}
	\\
	\headerrow
		{\emph{Freelancing Developer}}
		{\emph{02/2013 -- 10/2013}}
	\begin{itemize*}
		\item Created a cosine metric-based word sense disambiguation system leveraging text extracted from Wikipedia and DBpedia dumps; achieved performance comparable to the state-of-the-art.
	\end{itemize*}

\end{itemize}


\hrule
\vspace{-0.4em}
\subsection*{Education}

\begin{itemize}
	\parskip=0.1em
	
	\item 
	\headerrow
		{\textbf{National University of Ireland}}
		{\textbf{Galway, Ireland}}
	\\
	\headerrow
		{\emph{College of Engineering and Informatics, Ph.D. Natural Language Processing}}
		{\emph{10/2015 -- Present}}
	\begin{itemize*}
		\item I am interested in creating methods that allow efficient adaptation to novel domains and tasks in real-world scenarios. My research areas are domain adaptation, transfer learning, and multi-task learning for Natural Language Processing.
	\end{itemize*}
	
	\item 
	\headerrow
		{\textbf{University of Copenhagen}}
		{\textbf{Copenhagen, Denmark}}
	\\
	\headerrow
		{\emph{Natural Language Processing Group, Department of Computer Science}}
		{\emph{04/2017 -- 06/2017}}
	\begin{itemize*}
		\item Research visit invited by Anders S\o gaard.
		\item Research on multi-task learning, cross-lingual and cross-domain learning.
	\end{itemize*}

	\item 
	\headerrow
		{\textbf{Ruprecht-Karls-Universität Heidelberg}}
		{\textbf{Heidelberg, Germany}}
	\\
	\headerrow
		{\emph{Institute of Computational Linguistics, B.A. Computational Linguistics, English Linguistics}}
		{\emph{10/2012 -- 09/2015}}
	\begin{itemize*}
		\item Final grade: 1.0 (\href{https://en.wikipedia.org/wiki/Academic_grading_in_Germany}{German scale}), i.e. GPA 4.0; thesis: \emph{Construction and Analysis of an Emotion Proposition Store}
		\item Relevant courses: Statistics, Algorithms and Data Structures, Machine Learning, Formal Syntax \& Semantics
		\item Relevant online courses: Machine Learning (Stanford), AI (MIT), Into to Algorithms (Berkeley), Deep Learning for NLP (Stanford), Deep Learning (Oxford)
	\end{itemize*}
	
		\item 
	\headerrow
		{\textbf{Trinity College}}
		{\textbf{Dublin, Ireland}}
	\\
	\headerrow
		{\emph{School of Computer Science and Statistics, Computer Science and Language}}
		{\emph{09/2014 -- 01/2015}}
	\begin{itemize*}
		\item Semester abroad
		\item relevant courses: AI, Fuzzy Logic, High-Tech Entrepreneurship
	\end{itemize*}

\end{itemize}

\hrule
\vspace{-0.4em}
\subsection*{Awards}

\begin{itemize}
	\parskip=0.1em
	
	\item 
	\headerrow
		{Scholarship of the Irish Research Council}
		{\emph{10/2015 -- Present}}
	\item 
	\headerrow
		{Cusanuswerk scholarship of the German state}
		{\emph{04/2014 -- 09/2015}}	
	\item 
	\headerrow
		{Microsoft Certified Professional (Programming in C\#)}
		{\emph{06/2015}}
	\item 
	\headerrow
		{Best Delegate award in various Model United Nations conferences}
		{\emph{11/2012 -- 01/2014}}

\end{itemize}

\hrule
\vspace{-0.4em}
\subsection*{Languages and Technologies}

\begin{indentsection}{\parindent}
\hyphenpenalty=1000
\begin{description*}
	\item[Programming Languages:]
	Python, Java, C\#, R, C, \LaTeX, Prolog, JavaScript, SPARQL
	\item[Technologies:]
	SciPy, NumPy, Keras, TensorFlow, DyNet, scikit-learn, NLTK, CoreNLP, MALLET, Weka, UNIX, Git
	\item[Natural Languages:]
	Fluent in German and English, advanced in French and Spanish, beginner in Portuguese and Latin	
	\item[Open Source Contributions:]
	The OpenCog Foundation
\end{description*}
\end{indentsection}

\hrule
\vspace{-0.4em}
\subsection*{Other activities}

\begin{itemize}
	\parskip=0.1em
		\item 
	\headerrow
		{\textbf{Natural Language Processing Dublin organizer}}
		{\emph{08/2016 -- Present}}
	\begin{itemize*}
		\item Organized 7 events so far. Meetup\footnote{\url{https://www.meetup.com/NLP-Dublin/}} has 450+ members and connects students, researchers, and industry professionals.
	\end{itemize*}

\end{itemize}

\hrule
\vspace{-0.4em}
\subsection*{Publications}

\begin{enumerate}
	\parskip=0.1em
	
	\item Sebastian Ruder, Barbara Plank (2017). Learning to select data for transfer learning with Bayesian Optimization. In \textit{Proceedings of the 2017 Conference on Empirical Methods in Natural Language Processing}, Copenhagen, Denmark.
	
	\item Sebastian Ruder (2017). \href{https://arxiv.org/abs/1706.05098}{An Overview of Multi-Task Learning in Deep Neural Networks}. arXiv preprint arXiv:1706.05098.

	\item Sebastian Ruder (2017). \href{https://arxiv.org/abs/1706.04902}{A survey of cross-lingual embedding models}. arXiv preprint arXiv:1706.04902.
	
	\item Sebastian Ruder, Joachim Bingel, Isabelle Augenstein, Anders Søgaard (2017). \href{https://arxiv.org/abs/1705.08142}{Sluice networks: Learning what to share between loosely related tasks}. arXiv preprint arXiv:1705.08142.
	
	\item Sebastian Ruder, Parsa Ghaffari, John G. Breslin (2017). \href{https://arxiv.org/abs/1702.02426}{Data Selection Strategies for Multi-Domain Sentiment Analysis}. arXiv preprint arXiv:1702.02426.
	
	\item Sebastian Ruder, Parsa Ghaffari, John G. Breslin (2017). \href{https://arxiv.org/abs/1702.02052}{Knowledge Adaptation: Teaching to Adapt}. arXiv preprint arXiv:1702.02052.
	
	\item Sebastian Ruder, Parsa Ghaffari, John G. Breslin (2016). Towards a continuous modeling of natural language domains. In \textit{Proceedings of EMNLP 2016 Workshop on Uphill Battles in Language Processing: Scaling Early Achievements to Robust Methods}, pages 53-57, Austin, Texas, US.
	
	\item Sebastian Ruder, Parsa Ghaffari, John G. Breslin (2016). A Hierarchical Model of Reviews for Aspect-based Sentiment Analysis. In \textit{Proceedings of the 2016 Conference on Empirical Methods in Natural Language Processing}, pages 999–1005, Austin, Texas, US.
	
	\item Ian D. Wood and Sebastian Ruder (2016). \href{http://gsi.dit.upm.es/esa2016/Proceedings-ESA2016.pdf}{Emoji as emotion tags for tweets}. In \textit{Emotion and Sentiment Analysis Workshop, LREC}, Portorož, Slovenia.
	
	\item Sebastian Ruder, Peiman Barnaghi, John G. Breslin (2016). Analysis and Applications of a Novel Corpus of Influencers on Twitter. In \textit{Twitter for Research Conference}, Galway, Ireland.
	
	\item Sebastian Ruder, Parsa Ghaffari, John G. Breslin (2016). \href{http://www.anthology.aclweb.org/S/S16/S16-1026.pdf}{INSIGHT-1 at SemEval-2016 Task 4: Convolutional Neural Networks for Sentiment Classification and Quantification}. In \textit{Proceedings of the 10th International Workshop on Semantic Evaluation (SemEval 2016)}, San Diego, US.
	
	\item Sebastian Ruder, Parsa Ghaffari, John G. Breslin (2016). \href{http://www.aclweb.org/anthology/S/S16/S16-1053.pdf}{INSIGHT-1 at SemEval-2016 Task 5: Convolutional Neural Networks for Multilingual Aspect-based Sentiment Analysis}. In \textit{Proceedings of the 10th International Workshop on Semantic Evaluation (SemEval 2016)}, San Diego, US.
	
	\item Sebastian Ruder (2016). \href{https://arxiv.org/pdf/1609.04747.pdf}{An overview of gradient descent optimization algorithms}. arXiv preprint arXiv:1609.04747.

\end{enumerate}

%\hrule
%\vspace{-0.4em}
%\subsection*{Reviewing}
%
%\begin{itemize}
%	\parskip=0.1em
%	
%	\item 
%	\headerrow
%		{Reviewer SemEval-2016 Task 5: Aspect-based Sentiment Analysis}
%
%\end{itemize}

\hrule
\vspace{-0.4em}
\subsection*{Talks}

\begin{itemize}
	\parskip=0.1em
	
	\item Natural Language Processing Copenhagen Meetup Talk, May 2017: Transfer Learning for NLP\footnote{\url{https://www.slideshare.net/SebastianRuder/transfer-learning-for-natural-language-processing}}
	
	\item Accenture Tech Talk, March 2017: Transfer Learning -- The Next Frontier for Machine Learning
	
	\item LinkedIn Tech Talk, March 2017: Transfer Learning -- The Next Frontier for Machine Learning\footnote{\url{https://www.slideshare.net/SebastianRuder/transfer-learning-the-next-frontier-for-machine-learning}}
	
	\item NLP Dublin meetup, December 2016: NIPS 2016 Highlights\footnote{\url{http://www.slideshare.net/SebastianRuder/nips-2016-highlights-sebastian-ruder}}
	
	\item INSIGHT SIG NLP meetup, August 2016: A Hierarchical Model of Reviews for Aspect-based Sentiment Analysis\footnote{\url{http://www.slideshare.net/SebastianRuder/a-hierarchical-model-of-reviews-for-aspectbased-sentiment-analysis}}
	
	\item NLP Dublin meetup, August 2016: Softmax Approximations for Learning Word Embeddings and Language Modelling\footnote{\url{http://www.slideshare.net/SebastianRuder/softmax-approximations-for-learning-word-embeddings-and-language-modeling-sebastian-ruder}}
	
\end{itemize}

\end{document}

